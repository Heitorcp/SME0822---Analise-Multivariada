% Options for packages loaded elsewhere
\PassOptionsToPackage{unicode}{hyperref}
\PassOptionsToPackage{hyphens}{url}
%
\documentclass[
]{article}
\usepackage{amsmath,amssymb}
\usepackage{lmodern}
\usepackage{iftex}
\ifPDFTeX
  \usepackage[T1]{fontenc}
  \usepackage[utf8]{inputenc}
  \usepackage{textcomp} % provide euro and other symbols
\else % if luatex or xetex
  \usepackage{unicode-math}
  \defaultfontfeatures{Scale=MatchLowercase}
  \defaultfontfeatures[\rmfamily]{Ligatures=TeX,Scale=1}
\fi
% Use upquote if available, for straight quotes in verbatim environments
\IfFileExists{upquote.sty}{\usepackage{upquote}}{}
\IfFileExists{microtype.sty}{% use microtype if available
  \usepackage[]{microtype}
  \UseMicrotypeSet[protrusion]{basicmath} % disable protrusion for tt fonts
}{}
\makeatletter
\@ifundefined{KOMAClassName}{% if non-KOMA class
  \IfFileExists{parskip.sty}{%
    \usepackage{parskip}
  }{% else
    \setlength{\parindent}{0pt}
    \setlength{\parskip}{6pt plus 2pt minus 1pt}}
}{% if KOMA class
  \KOMAoptions{parskip=half}}
\makeatother
\usepackage{xcolor}
\usepackage[margin=1in]{geometry}
\usepackage{graphicx}
\makeatletter
\def\maxwidth{\ifdim\Gin@nat@width>\linewidth\linewidth\else\Gin@nat@width\fi}
\def\maxheight{\ifdim\Gin@nat@height>\textheight\textheight\else\Gin@nat@height\fi}
\makeatother
% Scale images if necessary, so that they will not overflow the page
% margins by default, and it is still possible to overwrite the defaults
% using explicit options in \includegraphics[width, height, ...]{}
\setkeys{Gin}{width=\maxwidth,height=\maxheight,keepaspectratio}
% Set default figure placement to htbp
\makeatletter
\def\fps@figure{htbp}
\makeatother
\setlength{\emergencystretch}{3em} % prevent overfull lines
\providecommand{\tightlist}{%
  \setlength{\itemsep}{0pt}\setlength{\parskip}{0pt}}
\setcounter{secnumdepth}{-\maxdimen} % remove section numbering
\ifLuaTeX
  \usepackage{selnolig}  % disable illegal ligatures
\fi
\IfFileExists{bookmark.sty}{\usepackage{bookmark}}{\usepackage{hyperref}}
\IfFileExists{xurl.sty}{\usepackage{xurl}}{} % add URL line breaks if available
\urlstyle{same} % disable monospaced font for URLs
\hypersetup{
  pdftitle={Lista01},
  pdfauthor={Heitor Carvalho Pinheiro - 11833351},
  hidelinks,
  pdfcreator={LaTeX via pandoc}}

\title{Lista01}
\author{Heitor Carvalho Pinheiro - 11833351}
\date{2022-09-15}

\begin{document}
\maketitle

\hypertarget{uxaa-lista-de-exercuxedcios---sme0822}{%
\subsection{1ª Lista de exercícios -
SME0822}\label{uxaa-lista-de-exercuxedcios---sme0822}}

\hypertarget{exercuxedcio-1.1.4}{%
\subsubsection{Exercício 1.1.4}\label{exercuxedcio-1.1.4}}

\begin{verbatim}
## # A tibble: 10 x 4
##    companies           sales profits assets
##    <chr>               <dbl>   <dbl>  <dbl>
##  1 Citigroup           108.    17.0   1484.
##  2 General Electric    152.    16.6    750.
##  3 American Intl Group  95.0   10.9    766.
##  4 Bank of America      65.4   14.1   1110.
##  5 HSBC Group           63.0    9.52  1031.
##  6 ExxonMobil          264.    25.3    195.
##  7 Royal Dutch/Shell   265.    18.5    194.
##  8 BP                  285.    15.7    191.
##  9 ING Group            92.0    8.1   1175.
## 10 Toyota Motor        166.    11.1    211.
\end{verbatim}

\begin{enumerate}
\def\labelenumi{\alph{enumi})}
\tightlist
\item
  Plot the scatter diagram and marginal dot diagrams for variables x1
  and x2. Comment on the appearance of the diagrams.
\end{enumerate}

\begin{verbatim}
## Warning: Using size for a discrete variable is not advised.
\end{verbatim}

\includegraphics{lista01_files/figure-latex/unnamed-chunk-2-1.pdf}

O gráfico marginal do lucro concentra-se em valores abaixo de 18
bilhões. Já o gráfico marginal de vendas, apresenta-se muito mais
disperso, com valores que variam de 10 bilhões a mais de 250 bilhões.

O gráfico de dispersão, visualmente, parece desmonstrar uma correlação
positiva entre as variáveis, porém não muito acentuada. Ou seja,
conforme as vendas aumentam o lucro também aumenta.

\begin{enumerate}
\def\labelenumi{\alph{enumi})}
\setcounter{enumi}{1}
\tightlist
\item
  Compute \(\bar{x}_{1}\), \(\bar{x}_{2}\), \(s_{11}\), \(s_{22}\),
  \(s_{1,2}\) and \(r_{12}\). Interprete \(r_{12}\)
\end{enumerate}

\begin{verbatim}
## [1] "Media x1 = 155.6"
## [1] "Media x2 = 14.7"
## [1] "Variancia x1 = 7476.45"
## [1] "Variancia x2 = 26.19"
## [1] "Covariancia = 303.62"
## [1] "Correlacao = 0.69"
\end{verbatim}

O valor de \(r_{12} = 0.69\), confirma a correlação positiva entre as
duas variáveis, porém, a correlação não é muito acentuada.

\hypertarget{exercuxedcio-1.1.5}{%
\subsubsection{Exercício 1.1.5}\label{exercuxedcio-1.1.5}}

Use the data in 1.1.4

\begin{enumerate}
\def\labelenumi{\alph{enumi})}
\tightlist
\item
  Plot the scatter diagrams and dot diagrams for (\(x_{2}, x_{3}\)) and
  (\(x_{1}, x_{3}\)) Comment on the patterns.
\end{enumerate}

Diagrama de dispersão para (\(x_{2}, x_{3}\))

\begin{verbatim}
## Warning: Using size for a discrete variable is not advised.
\end{verbatim}

\includegraphics{lista01_files/figure-latex/unnamed-chunk-4-1.pdf}

Não parece existir correlação evidente entre o lucro e valor em ações
das empresas. Semelhante aos lucros, o valor em ações de cada empresa,
possui grande variância, indo de empresas com duas centenas de bilhão em
ações a empresas com mais de trilhões em ações.

\begin{enumerate}
\def\labelenumi{\alph{enumi})}
\setcounter{enumi}{1}
\tightlist
\item
  Compute the \(\bar{x}\), \(S_{n}\) and \(\textbf{R}\) arrays for
  \((x_{1}, x_{2}, x_{3})\)
\end{enumerate}

\begin{verbatim}
## [1] 155.603
\end{verbatim}

\begin{verbatim}
## [1] 14.704
\end{verbatim}

\begin{verbatim}
## [1] 710.911
\end{verbatim}

\textbf{Vetor de Médias}

\$\$ \textbf{X} =

\begin{bmatrix}

155.6 \\

14.7 \\

711 \\

\end{bmatrix}

\$\$

\textbf{Vetor de Covariâncias}

\$\$ S\_\{n\} =

\begin{bmatrix}

7476.4 & 303.6 & -35576 \\

303.6 & 26.2 & -1053.9 \\

-35576 & -1053.9 & 23754.3 \\

\end{bmatrix}

\$\$

\textbf{Vetor de Correlação}

\$\$ \textbf{R} =

\begin{bmatrix}

1 & 0.69 & -0.84 \\

0.69 & 1 & -0.42 \\

-0.84 & -0.42 & 1 \\

\end{bmatrix}

\$\$

\hypertarget{exercise-1.1.24}{%
\subsubsection{Exercise 1.1.24}\label{exercise-1.1.24}}

\begin{verbatim}
## Warning: NAs introduced by coercion
\end{verbatim}

\begin{verbatim}
## Warning in snapshot3d(scene = x, width = width, height = height): webshot = TRUE
## requires the webshot2 package and Chrome browser; using rgl.snapshot() instead
\end{verbatim}

\includegraphics[width=6.5in]{C:\Users\heito\AppData\Local\Temp\RtmpuAOjQ4\file379858a96193}
\textbf{Checking for Outliers}

A análise do Scatterplot em 3D, nos permite identificar 2 outliers
facilmente.

\begin{itemize}
\tightlist
\item
  Outlier 1: \((x_1 = 0.11, x_2 = 1.74, x_3=2.49)\)
\item
  Outlier 2: \((x_1 = 0.66, x_2 = 11.05, x_3=2.32)\)
\end{itemize}

Também existem métodos quantitativos para detecção de outliers, os quais
não abordaremos aqui.

\hypertarget{exercuxedcio-03}{%
\subsubsection{Exercício 03}\label{exercuxedcio-03}}

\begin{enumerate}
\def\labelenumi{\alph{enumi})}
\tightlist
\item
  Avalie se há associação entre as variáveis faixa etária e exame
  solicitado.
\end{enumerate}

\textbf{Verificando a existência de dependência}

\begin{verbatim}
## 
##  
##    Cell Contents
## |-------------------------|
## |                       N |
## | Chi-square contribution |
## |           N / Row Total |
## |-------------------------|
## 
##  
## Total Observations in Table:  1110 
## 
##  
##                   | blood2$`Faixa Etária` 
##      blood2$Exame |    18 a 30 anos |    30 a 45 anos | 45 anos ou mais |     Ate 18 anos |       Row Total | 
## ------------------|-----------------|-----------------|-----------------|-----------------|-----------------|
## Eletrocardiograma |               6 |              20 |              60 |               6 |              92 | 
##                   |           3.976 |           1.262 |           5.192 |           0.632 |                 | 
##                   |           0.065 |           0.217 |           0.652 |           0.065 |           0.083 | 
## ------------------|-----------------|-----------------|-----------------|-----------------|-----------------|
##        Endoscopia |              24 |             100 |              80 |              10 |             214 | 
##                   |           1.520 |          27.086 |           5.583 |           4.466 |                 | 
##                   |           0.112 |           0.467 |           0.374 |           0.047 |           0.193 | 
## ------------------|-----------------|-----------------|-----------------|-----------------|-----------------|
##            Outros |              30 |              40 |              50 |              24 |             144 | 
##                   |           4.116 |           0.001 |           5.741 |           9.373 |                 | 
##                   |           0.208 |           0.278 |           0.347 |           0.167 |           0.130 | 
## ------------------|-----------------|-----------------|-----------------|-----------------|-----------------|
##            Sangue |             100 |             150 |             350 |              60 |             660 | 
##                   |           0.249 |           6.392 |           2.605 |           0.005 |                 | 
##                   |           0.152 |           0.227 |           0.530 |           0.091 |           0.595 | 
## ------------------|-----------------|-----------------|-----------------|-----------------|-----------------|
##      Column Total |             160 |             310 |             540 |             100 |            1110 | 
## ------------------|-----------------|-----------------|-----------------|-----------------|-----------------|
## 
## 
\end{verbatim}

Analisando as proporções marginais em relação às linhas, caso houvesse
independência entre as variáveis para cada elemento esperaríamos uma
proporção semelhante ao total da linha. Por exemplo, a proporção de
eletrocardiogramas corresponde a cerca de 8.3\% dos exames. Percebe-se
que conforme a idade aumenta a proporção do exame também aumenta e esse
padrão se observa entre todos os exames. Desse modo, \textbf{há
associação entre as variáveis faixa etária e exame solicitado}.

\begin{enumerate}
\def\labelenumi{\alph{enumi})}
\setcounter{enumi}{1}
\tightlist
\item
  Determine e interprete o coeficiente de Tschuprow
\end{enumerate}

Podemos definir o coeficiente de Tschuprow, a apartir da seguinte
formula:

\[
{\displaystyle \phi ^{2}=\sum _{i=1}^{r}\sum _{j=1}^{c}{\frac {(\pi _{ij}-\pi _{i+}\pi _{+j})^{2}}{\pi _{i+}\pi _{+j}}}}
\]

Em que \(\phi\) é o coeficiente de contigência da média quadrática.

E \({\displaystyle \pi _{i+}=\sum _{j=1}^{c}\pi _{ij}}\) é o somatório
das proporções para a linha \(i\) e
\({\displaystyle \pi _{+j}=\sum _{i=1}^{r}\pi _{ij}.}\) é o somatório
das proporções para a coluna \(j\).

Sendo assim, define-se o coeficiente de Tschuprow, da seguinte maneira.

\[
{\displaystyle T={\sqrt {\frac {\phi ^{2}}{\sqrt {(r-1)(c-1)}}}}.}
\]

Perceba que o coeficiente de Tschuprow é uma espécie de normalização do
valor de \(\phi\).

O coeficiente de Tschuprow \(T\) nos indica se existe correlação entre
variáveis categóricas em uma tabela de contigência.

Desse modo, um \(T>0\) indica que existe correlação, se \(T = 0\), não
existe correlação. Sendo \(0 \leq T \leq1\)

O coeficiente de Tschuprow para a tabela de contingência entre Faixa
Etária e Exame é:

\[
T = 0.15
\]

\begin{verbatim}
##                    Faixa Etária
## Exame               18 a 30 anos 30 a 45 anos 45 anos ou mais Ate 18 anos
##   Eletrocardiograma            6           20              60           6
##   Endoscopia                  24          100              80          10
##   Outros                      30           40              50          24
##   Sangue                     100          150             350          60
\end{verbatim}

\begin{verbatim}
## [1] "integer"
\end{verbatim}

\begin{verbatim}
## [1] 0.1532396
\end{verbatim}

O que implica que existe dependência entre as variáveis.

\hypertarget{exercuxedcio-6}{%
\subsubsection{Exercício 6}\label{exercuxedcio-6}}

Considerando a nossa forma quadrática
\(2x_1^2-2x_1x_2+x_2^2+4x_1x_3-3x_3^2\), podemos verificar que os
valores correspondentes à diagonal principal serão 2,1 e -3. Os valores
correspondentes às entradas \((i,j) = (1,2)\) e \((i,j) = (2,1)\) devem
resultar em -2, uma vez que temos \(-2x_1x_2\), logo, ambos são -1. Por
fim o mesmo princípio se aplica às entradas correspodentes a
\((i,j) = (1,3)\) e \((i,j) = (3,1)\), como temos que \(4x_1x_3\), logo
essas entradas recebem o valor 2. As entradas para \((i,j) = (2,3)\) e
\((i,j) = (3,2)\) recebem o valor 0.

Por fim, a matriz \(A\) associada à forma quadrática é:

\$\$ A =

\begin{bmatrix}

2 & -1 & 2 \\

-1 & 1 & 0 \\

2 & 0 & -3 \\

\end{bmatrix}

\$\$

\hypertarget{exercuxedcio-7}{%
\subsubsection{Exercício 7}\label{exercuxedcio-7}}

\end{document}
